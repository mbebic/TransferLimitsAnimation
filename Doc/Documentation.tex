% ******************************************************************************
% File name: Documentation.tex
% Title:   Documentation for Transfer Limits Animation
% Subject: Documentation based on refercng portions of listing
%
% Created: June 11, 2017
% Original Issue: Pending
% Last Edited: June 11, 2017
% Latest Rev. #: 0
%
% Ref. Docs.:
%
% Revision Notes:
% Rev.0 (June 11/2017 by JZB) Template
% ******************************************************************************
%
%        1         2         3         4         5         6         7         8
% 345678901234567890123456789012345678901234567890123456789012345678901234567890
%

\documentclass[12pt]{report}

% List of all packages is included here
\input{Packages.tex}

\renewcommand\lstlistlistingname{List of Listings}

\makeindex

\title{Transfer Limits Animation}
\author{Jovan Bebi\'{c}\\[0.5ex]
        Marija Bebi\'{c}}

\begin{document}
\pagenumbering{roman} % Changes page numbering to roman
\maketitle

\begin{abstract}
This document is a LaTeX template for creating code documentation. It is typeset
\emph{report} document class, which means that it has chapters and sections.
The main purpose of the document is to illustrates use of figures and listings,
but it also showcases cross referencing bibliography entries, etc.
\end{abstract}

\pagenumbering{arabic} % Returns to arabic page numbering
\section*{Revision Notes}
\begin{tabularx}{\textwidth}{|p{1in}|l|c|X|}
\hline
Authors & Date & Rev & Notes \\
\hline
JZB & Jun 11-14, 2017 & Draft & Template for use by others \\
\hline
\end{tabularx}

\tableofcontents
\listoffigures
\addcontentsline{toc}{chapter}{\lstlistlistingname}{\lstlistoflistings}
\lstlistoflistings

\doublespacing
\chapter{Introduction}
\label{ch:intro}
This is a template LaTeX file useful when preparing software documentation. It
is compiled with pdflatex using included DOS batch file called runAll.bat. In the following
\section{Including and captioning figures}
\label{sec:figures}
An example figure is shown in Fig.~\ref{fig:1}.

\begin{figure*}%[t]
  \centering %
  \includegraphics[scale=0.8, page=1]{visuals/Plots1}
  \caption[Area markers] {Area markers arranged on a circle} %
  \label{fig:1}
\end{figure*}

\clearpage

\section{Including and anotating code}
\label{sec:code}

\singlespacing
\lstinputlisting[language=python,caption={\lstname},label=lst:1, linerange={85-95}]{code/base_frame.py}
\doublespacing

\section{Other formatting}

\singlespacing
\begin{enumerate}
  \item Preprocess historic AMI data to enable study based on actual measurements.
  \item Execute OpenDSS in snapshot mode to review voltage contours at a wide
        range of operating conditions.
  \item Review the resulting patterns and place OpenDSS \emph{monitors} at
        nodes that experience greatest voltage change.
  \item Execute OpenDSS in temporal mode to collect temporal voltage recordings
        at monitored nodes.
  \item Review voltage histograms at monitored buses to quantify the frequency
        and magnitude of voltage excursions.
\end{enumerate}
\doublespacing

In the following sections, we briefly describe the tool-chains that facilitate
this process.

\clearpage

\bibliographystyle{IEEEtran} % "plain" is another option
% \bibliography{Ph2report_EC}

\clearpage

\end{document}
